\documentclass[draft, 11pt, a4paper,twocolumn]{article}
\usepackage{times}
\usepackage[utf8]{inputenc}
\usepackage[IL2]{fontenc}
\usepackage[czech]{babel}

\usepackage[unicode, hidelinks]{hyperref}
\usepackage[left=1.5cm,text={18cm, 25cm},top=2.5cm]{geometry}
\usepackage{amsthm, amsmath, amssymb}


\newtheorem{definition}{Definice}
\newtheorem{sentence}{Věta}
\renewcommand\labelitemi{$\bullet$}


\begin{document}

\begin{titlepage}
    \begin{center}
        \Huge
        \textsc{Fakulta informačních technologií}\\
        \textsc{Vysoké učení technické v~Brně}\\
        \vspace{\stretch{0.382}}
        \LARGE
        Typografie a publikování\,--\,2. projekt\\
        Sazba dokumentů a matematických výrazů\\
        \vspace{\stretch{0.618}}
    \end{center}
    {\Large 2021 \hfill Jiří Mládek (xmlade01)}
\end{titlepage}


\section*{Úvod}
V~této úloze si vyzkoušíme sazbu titulní strany, matematic\-kých vzorců, prostředí a dalších textových struktur obvyklých pro technicky zaměřené texty (například rovnice \eqref{eq:1}
nebo Definice \ref{def:1} na straně \pageref{def:1}). Rovněž si vyzkoušíme používání odkazů \verb|\ref| a \verb|\pageref|.

Na titulní straně je využito sázení nadpisu podle optického středu s~využitím zlatého řezu. Tento postup byl probírán na přednášce. Dále je použito odřádkování se zadanou relativní velikostí 0.4 em a 0.3 em.

V~případě, že budete potřebovat vyjádřit matematickou konstrukci nebo symbol a nebude se Vám dařit jej nalézt v~samotném \LaTeX u, doporučuji prostudovat možnosti balíku maker \AmS -\LaTeX.

\section{Matematický text}
Nejprve se podíváme na sázení matematických symbolů ~ a výrazů v~plynulém textu včetně sazby definic a vět s~využitím balíku \verb|amsthm|. Rovněž použijeme poznámku pod čarou s~použitím příkazu \verb|\footnote|. Někdy je vhodné použít konstrukci \verb|\mbox{}|, která říká, že text nemá být zalomen.

\begin{definition}\label{def:1}
\emph{Rozšířený zásobníkový automat} (RZA) je definován jako sedmice tvaru $A=(Q,\Sigma,\Gamma,\delta,q_0,Z_0,F)$, kde:
    \begin{itemize}
        \item Q je konečná množina \emph{vnitřních (řídicích) stavů},
        \item $\Sigma$ je konečná \emph{vstupní abeceda},
        \item $\Gamma$ je konečná \emph{zásobníková abeceda},
        \item $\delta$ je \emph{přechodová funkce} $Q \times (\Sigma\cup\{\epsilon\}) \times \Gamma^* \rightarrow 2^{Q\times\Gamma^*}$,
        \item $q_0 \in Q$ je \emph{počáteční stav}, $Z_0 \in \Gamma$ \emph{je startovací symbol zásobníku} a  $F \subseteq Q$ je množina \emph{koncových stavů}.
    \end{itemize}
    
    \emph{Nechť} $P=(Q,\Sigma,\Gamma,\delta,q_0,Z_0,F)$ \emph{je rozšířený zásobníkový automat}. Konfigurací \emph{nazveme trojici} $(q,w,\alpha)\in Q\times\Sigma^*\times\Gamma^*$, \emph{kde} $q$ \emph{je aktuální stav vnitřního řízení}, $w$ \emph{je dosud nezpracovaná část vstupního řetězce a} $\alpha=Z_{i_1}Z_{i_2}\ldots Z_{i_k}$ \emph{je obsah zásobníku\footnote{$Z_{i_1}$ je vrchol zásobníku}}.
\end{definition}

\subsection{Podsekce obsahující větu a odkaz}
    \begin{definition}\label{def:2}
    \emph{Řetězec} $w$ \emph{nad abecedou} $\Sigma$ \emph{je přijat RZA} $A$ jestliže $(q_0, w, Z_0)\overset{*}{\underset{A}\vdash}(q_F, \epsilon, \gamma)$ pro nějaké $\gamma \in \Gamma^*$ a $q_F \in F$. Množinu $L(A) = \{w\hspace{2mm}|\hspace{2mm}w$ je přijat RZA A\} $\subseteq \Sigma^*$ nazýváme \emph{jazyk přijímaný RZA} $A$.
    \end{definition}
    
    Nyní si vyzkoušíme sazbu vět a důkazů opět s~použitím balíku \verb|amsthm|.
    \begin{sentence}
    Třída jazyků, které jsou přijímány ZA, odpovídá \emph{bezkontextovým jazykům}.
    \end{sentence}
    \begin{proof}
        V~důkaze vyjdeme z~Definice \ref{def:1} a \ref{def:2}.
    \end{proof}
    

\section{Rovnice a odkazy}
Složitější matematické formulace sázíme mimo plynulý text. Lze umístit několik výrazů na jeden řádek, ale pak je třeba tyto vhodně oddělit, například příkazem \verb|\quad|.

$$\sqrt[i]{x_i^3}\quad\text{kde }x_i\text{ je }i\text{-té sudé číslo splňující}\quad x_i^{x_i^{i^2}+2} \leq y_i^{x_i^4}$$

V~rovnici \eqref{eq:1} jsou využity tři typy závorek s~různou explicitně definovanou velikostí.
\begin{eqnarray}
    x & = & \bigg[ \Big\{ \big[a+b\big]*c\Big\}^d\oplus 2\bigg]^{3/2}\label{eq:1}\\
    y & = & \lim_{x\to\infty} \frac{\frac{1}{\log_{10}x}}{\sin^2{x}+\cos^2{x}} \nonumber
\end{eqnarray}

V~této větě vidíme, jak vypadá implicitní vysázení limity $\lim_{n\to\infty}f(n)$ v~normálním odstavci textu. Podobně je to i s~dalšími symboly jako $\prod _{i=1}^n 2^i$ či $\bigcap _{A \in \mathcal{B}}A$. V~případě vzorců $\lim\limits _{n\to\infty}f(n)$ a $\prod\limits _{i=1}^n 2^i$ jsme si vynutili méně úspornou sazbu příkazem \verb|\limits|.
\begin{equation}
    \int_b^a g(x)\,\mathrm{d}x \quad=\quad -\int\limits _a^b f(x)\,\mathrm{d}x
\end{equation}


\section{Matice}
Pro sázení matic se velmi často používá prostředí \verb|array| a závorky (\verb|\left|, \verb|\right|).

$$ \left( \begin{array}{ccc}
    a-b & \widehat{\xi + \omega}  & \pi \\
    \vec{\mathbf{a}} & \overleftrightarrow{AC} & \hat{\beta} \\
\end{array} \right)
=1 \Longleftrightarrow \mathcal{Q} = \mathbb{R}$$ 
$$
    \mathbf{A} = 
    \left\|
    \begin{array}{cccc}
    a_{11} & a_{12} & \ldots & a_{1n} \\
    a_{21} & a_{22} & \ldots & a_{2n} \\
    \vdots & \vdots & \ddots & \vdots \\
    a_{m1} & a_{m2} & \ldots & a_{mn} \\
    \end{array} \right\|
    =
    \left|
    \begin{array}{cc}
    t & u \\
    v & w \\
    \end{array} \right|
    = tw-uv
$$

Prostředí \verb|array| lze úspěšně využít i jinde.

$$
\begin{pmatrix}
    n \\
    k\\
\end{pmatrix}
=
\left\{ \begin{array}{c l}
0 & \text{pro } k < 0 \text{ nebo } k > n \\
\frac{n!}{k!(n-k)!} & \text{pro } 0 \leq k \leq n.
\end{array} \right.
$$

\end{document}