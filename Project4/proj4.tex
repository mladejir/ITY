\documentclass[11pt, a4paper]{article}
\usepackage{times}
\usepackage[left=2cm,text={17cm, 24cm},top=3cm]{geometry}
\usepackage[utf8]{inputenc}
\usepackage[IL2]{fontenc}
\usepackage[czech]{babel}
\usepackage{url}
\DeclareUrlCommand\url{\def\UrlLeft{<}\def\UrlRight{>} \urlstyle{tt}}
\usepackage[unicode, breaklinks]{hyperref}
\def\UrlBreaks{\do\/\do-}

\DeclareUnicodeCharacter{0301}{\'{e}}

\begin{document}

\begin{titlepage}
    \begin{center}
        \Huge
        \textsc{Vysoké učení technické v~Brně}\\
        \huge
        \textsc{Fakulta informačních technologií}\\
        \vspace{\stretch{0.382}}
        \LARGE
        Typografie a publikování\,--\,4. projekt\\
        \Huge
        Citace\\
        \vspace{\stretch{0.618}}
    \end{center}
    {\Large \today \hfill Jiří Mládek}
\end{titlepage}

\section{Webová typografie}

Pokud se chce vývojář webových stránek odlišit od konkurence, nesmí zapomínat ani na základní typografiká pravidla. Ta totiž u~uživatelů hrají nevědomě obrovskou roli. Hned první pohled na stránku uživatele buď upoutá, nebo naopak odradí. Konkrétněji jsou psychologické účinky na čtenáře popsány zde \cite{Grigerova2020}.

\subsection{Velikost písma}
Asi nejdůležitějším aspektem je velikost písma. Ta se pro web na desktopu pohybuje okolo 15-22 pixelů a pro mobil 14-16 pixelů. Dávejte si ale pozor, každý druh písma má jinou optickou velikost \cite{Marcikova2020}.

\subsection{Font}
Dále je důležité správně vybrat font písma. Pravidlo říká, že bychom měli nakombinovat 3 písmové rodiny: dobře čitelné serifové písmo, jednoduché bezserifové písmo a také titulkový font \cite{Saltz2010}.  K~doporučeným fontům patří font \emph{Verdana} a \emph{Georgia} \cite{Anonymous2004}. Toto souvisí se správným použitím nadpisů a podnadpisů, o~kterém se pojednává v~knize od Jaroslava Hrubého \cite{Hruby2003}.

\subsection{Řádek}
Pro lepší čitelnost je důležité mít správnou výšku řádku. Pro zachování vhodného odstupu mezi řádky je doporučeno, aby poměr výšky řádku a velikosti písma byl minimálně 1,4 \cite{Cecetka2010}.
Délka řádku by měla být závislá na velikosti obrazovky, je ale důležité myslet na to, že moc dlouhá délka řádku namáhá oči \cite{Kupferschmidt2015}.

\subsection{Kontrast}
Kvůli čitelnosti je také důležité mít dostatečný kontrast mezi písmem a pozadím. Zde platí pravidlo jednoduchosti, tedy černé písmo na bílém pozadí. Je také vhodné, aby čtenář ihned poznal, která část textu je důležitá. Toho docílíme použitím tučného písma \cite{Brda2016}.

\subsection{V~jednoduchosti je krása}
V~posledních letech se od nadbytečných prvků v~designu vracíme zpět k~jednoduchosti. Jedná se o~tzv. flat design, který se snaží o~co nejjednodušší rozhranní, odstraňuje vše nadbytečné \cite{Uhlirova2016}.
Tento přístup neplatí pouze pro web, ale například i pro knihy. Ve svém článku o~tom píše Martin Pecina. Popisuje mimo jiné, jaké vlastnosti je vhodné zvolit pro konkrétní typy knih \cite{Pecina2015}.


    \newpage
	\bibliographystyle{czechiso}
	\renewcommand{\refname}{Literatura}
	\bibliography{proj4}

\end{document}